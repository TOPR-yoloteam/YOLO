\section{Methodik}
\subsection{Softwarearchitektur}
\subsection{Kennzeichenerkennung}
\subsubsection{Testaufbau}
\subsubsection{Datensammlung und Annotation}
\subsubsection{Training und Optimierung des Modells}
\subsubsection{Herausforderungen bei realen Bedingungen}
\subsubsection{Optimierung für wechselnde Lichtverhältnisse und Kameraperspektiven}
\subsubsection{Evaluation und Genauigkeitsanalyse}

\subsection{Gesichtserkennung}
\subsubsection{Testaufbau}
\subsubsection{YOLO}
\paragraph{Vergleich der YOLO-Modelle}
\paragraph{Metriken}
\paragraph{Anwendungsbeispiel: Zutrittskontrollen}

\subsubsection{MediaPipe}
\paragraph{Einführung in MediaPipe}
MediaPipe ist ein Open-Source-Framework von Google, das die Entwicklung komplexer Computer-Vision-Anwendungen erleichtert. Es ermöglicht die Erstellung effizienter, plattformübergreifender Machine-Learning-Pipelines für die Verarbeitung von Video-, Bild- und Audiodaten in Echtzeit. Dank seiner modularen Architektur und der Nutzung vorgefertigter Komponenten, sogenannter "Calculators", können Entwickler schnell Prototypen erstellen und diese zu ausgereiften Anwendungen weiterentwickeln.  \\
Ein zentrales Merkmal von MediaPipe ist seine graphbasierte Architektur. Datenflüsse werden in sogenannten "Graphs" definiert, wobei jeder Knotenpunkt ("Node") spezifische Aufgaben übernimmt. Diese Struktur ermöglicht eine flexible und effiziente Verarbeitung von Datenströmen, was besonders für Anwendungen mit hohen Echtzeitanforderungen von Vorteil ist.  \\
MediaPipe bietet eine Vielzahl vortrainierter Lösungen, darunter die Gesichtserkennung und die Erkennung von Gesichtsmerkmalen (Face Landmarks). Diese Tools sind für den Einsatz auf verschiedenen Plattformen optimiert und können sowohl auf leistungsstarken Servern als auch auf mobilen Geräten in Echtzeit betrieben werden. \\
Im Kontext dieses Projekts wird MediaPipe insbesondere für die Gesichtserkennung und die Analyse von Gesichtsmerkmalen eingesetzt. Dies ermöglicht eine präzise und effiziente Identifikation von Personen, was beispielsweise bei Zutrittskontrollsystemen von großer Bedeutung ist.\\

\paragraph{Komponenten der Gesichtsanalyse}
\subparagraph{Gesichtserkennung}
Die Gesichtserkennung in MediaPipe basiert auf dem BlazeFace-Modell, einem speziell für mobile und eingebettete Geräte optimierten neuronalen Netzwerk. BlazeFace zeichnet sich durch seine hohe Effizienz und geringe Rechenlast aus, was eine schnelle und ressourcenschonende Gesichtserkennung in Echtzeit ermöglicht. Insbesondere auf Geräten mit begrenzter Rechenleistung wie Smartphones. Im Gegensatz zu umfassenderen Gesichtsanalysemodellen konzentriert sich BlazeFace primär auf die Detektion von Gesichtern, ohne dabei detaillierte Gesichtslandmarken zu berücksichtigen. \\
Es existieren mehrere Varianten von BlazeFace, die jeweils für unterschiedliche Anwendungsbereiche und Kameradistanzen optimiert wurden, wobei alle Versionen auf einem gemeinsamen, effizienten Basismodell aufbauen. Dieses Modell bietet einen ausgewogenen Kompromiss zwischen Genauigkeit und Geschwindigkeit. \\
Die Variante BlazeFace (short-range) ist für den Einsatz in Nahbereichsszenarien konzipiert, typischerweise bei einer Entfernung von bis zu zwei Metern zur Kamera. Sie eignet sich daher besonders für Selfie-Aufnahmen oder Interaktionen mit der frontseitigen Kamera mobiler Endgeräte. Diese Version ist die leichtgewichtigste und schnellste Ausführung des Modells und wurde explizit auf nahe Gesichter hin optimiert.
Demgegenüber ist BlazeFace (full-range) für einen erweiterten Entfernungsbereich von bis zu fünf Metern vorgesehen. Dieses Modell ist besser für die Verwendung mit rückseitigen Smartphone-Kameras oder anderen Szenarien geeignet, in denen sich die Zielperson weiter vom Aufnahmegerät entfernt befindet. Es bietet eine höhere Erkennungsgenauigkeit über verschiedene Distanzen hinweg, ist jedoch in der Regel rechenintensiver als das Short-Range-Modell. \\
Die Variante BlazeFace Sparse (full-range) stellt eine kompaktere Version des regulären Full-Range-Modells dar. Sie ist etwa 60\% kleiner, was sich in einer erhöhten Inferenzgeschwindigkeit und reduzierten Rechenkosten niederschlägt. Trotz ihrer geringeren Modellgröße bleibt sie für weiter entfernte Gesichter geeignet, kann jedoch in bestimmten Anwendungsfällen Einbußen in der Erkennungsgenauigkeit im Vergleich zur nicht-abgespeckten Full-Range-Version aufweisen. \\
Die Architektur von BlazeFace basiert auf einer effizienten Umsetzung von Convolutional Neural Networks (CNNs), wobei der Fokus auf der Reduktion von Modellparametern und Rechenoperationen liegt. Dadurch eignet sich das Modell ideal für den Einsatz in Echtzeitanwendungen auf mobilen Geräten. \\
Ein zentrales Konzept innerhalb der BlazeFace-Architektur ist der Einsatz sogenannter Ankerboxen. Dabei handelt es sich um eine Sammlung vordefinierter Begrenzungsrahmen unterschiedlicher Größen und Seitenverhältnisse, die systematisch über das Eingabebild verteilt werden. Für jede dieser Ankerboxen lernt das Modell, die Wahrscheinlichkeit zu bestimmen, ob sich ein Gesicht innerhalb des Rahmens befindet, und wie dieser Rahmen angepasst werden muss, um die tatsächliche Position und Größe des Gesichts möglichst präzise zu erfassen. Durch die Vielfalt an Formen und Größen dieser Ankerboxen kann das Modell Gesichter mit unterschiedlichen Dimensionen und Proportionen robust erkennen. \\
Der Erkennungsprozess von Gesichtern mit BlazeFace erfolgt in mehreren, aufeinanderfolgenden Schritten, die zusammen eine effiziente und präzise Identifikation von Gesichtern ermöglichen. Zu Beginn des Prozesses steht die Bildvorverarbeitung, bei der das Eingabebild auf eine einheitliche Größe, etwa 256x256 Pixel, skaliert wird. Diese Reskalierung ist notwendig, um die Eingabedaten in eine für das neuronale Netzwerk geeignete Form zu bringen und die Verarbeitung zu optimieren. Zusätzlich wird das Bild normalisiert, sodass die Pixelwerte in einem vordefinierten Bereich liegen, beispielsweise zwischen [0, 1] oder [-1, 1]. Dies stabilisiert das Training sowie die Inferenz des Netzwerks und beschleunigt den gesamten Erkennungsprozess. \\
Nach der Vorverarbeitung folgt die Merkmalsextraktion, bei der das Bild durch die verschiedenen Faltungsschichten des BlazeFace-Netzwerks geführt wird. Jede Schicht extrahiert zunehmend komplexere visuelle Merkmale, wie Kanten, Texturen und Formen, die für die spätere Gesichtserkennung entscheidend sind. Diese Merkmale helfen dem Modell, relevante Informationen aus dem Bild zu extrahieren, die für die Identifizierung von Gesichtern notwendig sind. \\ 
Im nächsten Schritt, den Vorhersageebenen, wird das Bild auf verschiedene vorab definierte Ankerboxen angewendet. Diese Ankerboxen repräsentieren potenzielle Bereiche im Bild, in denen ein Gesicht erkannt werden könnte. Für jede Ankerbox generiert das Netzwerk mehrere Ausgaben: Zunächst wird die Klassenwahrscheinlichkeit berechnet, die angibt, wie wahrscheinlich es ist, dass sich ein Gesicht innerhalb der Ankerbox befindet. Diese Wahrscheinlichkeit liegt zwischen 0 und 1 und stellt eine binäre Klassifikation dar (Gesicht oder kein Gesicht). Zudem wird eine Begrenzungsrahmen-Regression durchgeführt, bei der das Modell Offsets berechnet, um die Position und Größe der Ankerbox so anzupassen, dass sie das erkannte Gesicht möglichst genau umschließt. Zusätzlich wird eine Gesichtspunkt-Regression durchgeführt, bei der die Positionen der relevanten Gesichtspunkte, wie Augen, Nase, Mundwinkel und Ohransätze, relativ zur Ankerbox bestimmt werden. \\
Nach der Vorhersage folgt das Postprocessing, um die endgültigen Erkennungsergebnisse zu verfeinern. Zunächst wird eine Filtern nach Konfidenz durchgeführt, bei der Vorhersagen mit einer Konfidenzwahrscheinlichkeit unterhalb eines festgelegten Schwellenwerts verworfen werden. Dieser Schwellenwert kann je nach Anwendung angepasst werden, um die Balance zwischen der Anzahl der erkannten Gesichter und der Wahrscheinlichkeit von Fehlalarmen zu optimieren. Danach wird eine Non-Maximum Suppression (NMS) angewendet, um überlappende Begrenzungsrahmen, die dasselbe Gesicht erkennen, zu reduzieren. Nur der Rahmen mit der höchsten Konfidenz wird beibehalten, während andere mit niedrigerer Konfidenz verworfen werden. Dies stellt sicher, dass für jedes erkannte Gesicht nur eine eindeutige Erkennung ausgegeben wird. Abschließend werden die vorhergesagten Offsets verwendet, um die endgültigen Koordinaten der Begrenzungsrahmen und Gesichtspunkte relativ zum ursprünglichen Bild zu berechnen. Diese normalisierten Koordinaten werden in absolute Pixelkoordinaten umgewandelt, um die Gesichter und ihre Merkmale im Bild darzustellen. \\
Durch diesen mehrstufigen Erkennungsprozess kann BlazeFace eine schnelle und präzise Gesichtserkennung durchführen, die sowohl auf mobilen Geräten als auch in Echtzeitanwendungen effektiv eingesetzt werden kann.

\subparagraph{Gesichtspunkterkennung}
\subparagraph{Gesichtswiedererkennung mit Gesichtspunkte}
\paragraph{Metriken}
\paragraph{Anwendungsbeispiel: Zutrittskontrollen}

\subsubsection{Vergleich}
\paragraph{Confidence Score}
\paragraph{FPS / Inference Time}

\subsubsection{Manipulierte Gesichtserkennung}
\paragraph{Angriffsmethoden auf Gesichtserkennungssysteme}
\paragraph{Testen der Robustheit gegen Manipulationen}
\paragraph{Möglichkeiten zur Absicherung}

\subsubsection{Projektdokumentation}
\paragraph{Vorgehensmodell und Teamorganisation}
\paragraph{Dokumentation des Projektmanagements}