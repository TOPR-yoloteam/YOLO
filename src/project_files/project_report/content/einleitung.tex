\section{Einleitung}
\subsection{Motivation}
Gesichts- und Schrifterkennungstechnologien gehören heute zu den zentralen Komponenten moderner automatisierter Systeme. Anstatt nur visuelle Daten aufzunehmen, können Maschinen durch diese Technologien spezifische Muster, Objekte oder Zeichen erkennen und klassifizieren. Dieser technologische Fortschritt hat das Potenzial, die zukünftige Entwicklung in Schlüsselbereichen wie der Medizin und der Verteidigungstechnologie maßgeblich zu beeinflussen - etwa durch präzisere Diagnoseverfahren, automatisierte Auswertung medizinischer Bilddaten oder den Einsatz intelligenter Erkennungssysteme in sicherheitsrelevanten Anwendungen und militärischen Szenarien. 
Trotz dieser Erfolge stoßen aktuelle Systeme immer noch an ihre Grenzen. Faktoren wie schlechte Bildqualität, komplexe Lichtverhältnisse, verdeckte Gesichter oder unterschiedliche Schriftarten können die Erkennungsgenauigkeit erheblich beeinträchtigen. Zusätzlich werfen Fragen des Datenschutzes und der ethischen Nutzung neue Herausforderungen auf, die nicht ignoriert werden dürfen. Statt vorgefertigte Lösungen zu liefern, möchten wir Studierenden einen einfachen und praxisnahen Zugang zu komplexen Technologien wie der Gesichts- und Schrifterkennung ermöglichen. Ziel ist es, Neugier zu wecken und ein tieferes Verständnis für intelligente Systeme zu schaffen. 
\subsection{Zielsetzung des Projekts} 
\subsubsection{Primäres Ziel}Ziel dieses Projekts ist die Entwicklung eines praxisorientierten Hands-On Tutorials für internationale Studierende im Rahmen der THU Summer School. Im Fokus steht die Umsetzung grundlegender Methoden zur Personen- und Schrifterkennung unter kostengünstigen Laborbedingungen. Dabei kommt ein Raspberry Pi als preiswerte und kompakte Hardwareplattform zum Einsatz, die eine realistische Umsetzung auch unter begrenzten Ressourcen ermöglicht. 
Das Tutorial vermittelt nicht nur technische Grundlagen, sondern demonstriert anhand zweier konkreter Anwendungsbeispiele die Leistungsfähigkeit aktueller Computer-Vision-Technologien auf Basis der YOLOv5 und YOLOv8-Architektur:
\begin{enumerate}
    \item\textbf{Schrifterkennung (OCR):} Die automatische Erkennung und das Auslesen von Kfz-Kennzeichen aus Bildmaterial mithilfe der Zeichenerkennung. Ziel ist es, die alphanumerischen Informationen aus Bildern zu extrahieren.
    \item\textbf{Personenerkennung:} Die Erkennung und Identifikation von Personen über Gesichter in Bild- oder Videodaten. Neben der reinen Gesichtserkennung ermöglicht das System auch das Einlernen neuer Gesichter, sodass unbekannte Personen erfasst, gespeichert und bei späterem Auftreten wiedererkannt werden können.
\end{enumerate}
    Neben der Implementierung liegt ein besonderer Schwerpunkt auf der praktischen Anwendbarkeit in ressourcenbeschränkten Umgebungen sowie auf der kritischen Auseinandersetzung mit den Grenzen und ethischen Fragestellungen dieser Technologien. 
\subsubsection{Sekundäres Ziel}Neben der Umsetzung im Laborumfeld dienen beide Erkennungsaufgaben als praxisnahe Demonstrationen potenzieller Anwendungsszenarien. Dabei steht nicht die Entwicklung eines marktreifen Produkts im Vordergrund, sondern die schrittweise Hinführung zu einem Verständnis für den praktischen Einsatz solcher Technologien. 

Im Fall der Schrifterkennung wird ein automatisierter Mautprozess simuliert, bei dem Kfz-Kennzeichen erfasst und ausgewertet werden. Dieses Szenario vermittelt Einblicke in die Funktionsweise moderner Verkehrssysteme. 
Die Gesichtserkennung wird im Kontext einer Zutrittskontrolle umgesetzt, bei der autorisierte Personen anhand ihrer Gesichter identifiziert werden. Ziel ist es, exemplarisch aufzuzeigen, wie solche Systeme technisch realisiert werden können und welche Herausforderungen, in Bezug auf Genauigkeit oder Datenschutz, damit einhergehen. 

Unser Projekt versteht sich nicht als Endpunkt, sondern als Einladung, weiterzudenken, zu experimentieren und sowohl die Potenziale als auch Grenzen intelligenter Systeme kritisch und kreativ zu erforschen. 