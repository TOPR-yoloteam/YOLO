\section{Ergebnis}
\subsection{Zusammenfassung der Ergebnisse aus Nutzersicht}
Um die Praxistauglichkeit unseres Hands-On Tutorials zur Gesichts- und Schrifterkennung zu bewerten, haben wir Studierende der Technischen Hochschule Ulm (THU) sowie Studierende der Hochschule Neu-Ulm (HNU) eingeladen, das Tutorial selbstständig zu bearbeiten. Dabei stellten wir die notwendige Hardware zur Verfügung und leisteten bewusst nur minimale Hilfestellung, um ein möglichst realistisches Bild der Benutzerfreundlichkeit zu erhalten. Für die Bearbeitungszeit wählten wir einen Zeitraum von etwa dreieinhalb Stunden. Zwar ist für die Summer School ein Zeitraum von vier Stunden für die Bearbeitung vorgesehen, dennoch haben wir die Bearbeitungszeit für unseren Testdurchlauf 
gekürzt da wir auch Verspätungen oder Verzögerungen im Plan abdecken möchten. Das soll den Studenten die Chance zu geben das Tutorial vollständig abschließen zu können. In diesem Kapitel fassen wir die Rückmeldungen, Beobachtungen und Verbesserungsvorschläge der Teilnehmenden zusammen. \singlespacing
Ein Studierender (Studiengang: BWL) ohne formale Ausbildung in einem technischen Studiengang, jedoch mit ersten Erfahrungen in Webentwicklung, bearbeitete das Tutorial. Er berichtete, dass die grundlegende Konfiguration des Raspberry Pi für ihn überraschend reibungslos verlief, was er insbesondere auf die Bereitstellung eines klar strukturierten Installationsskripts zurückführte. Besonders die erste Verbindung zur Hardware und das Starten des Systems verliefen für ihn problemlos, da das Skript es ermöglichte, die notwendigen Schritte sicher und ohne größere Fehlermöglichkeiten auszuführen. Die eigentliche Herausforderung bestand für ihn darin, sich als jemand ohne tiefere Kenntnisse in Python-Programmierung sowie in der Bearbeitung von Bildmaterial zurechtzufinden, insbesondere im Abschnitt zur Erkennung und Auswertung von Kfz-Kennzeichen. Mit Mühe und kleinen Hilfestellungen von unserer Seite gelang es ihm, fast das vollständige Tutorial innerhalb des vorgesehenen Bearbeitungszeitraums von etwa drei bis vier Stunden abzuschließen. Zwei weitere Studenten (Studiengänge: Information Management Automotive, Informationsmanagement und Unternehmenskommunikation), die ebenfalls keinen technischen Hintergrund haben und sich im Gegensatz zum ersten Teilnehmer nicht mit Webentwicklung auseinandergesetzt haben, bearbeiteten gemeinsam dieses Tutorial. Auch hier war das Einrichten des Raspberry Pi kein Problem, die Aufgaben zur Kennzeichen- und Personenerkennung jedoch herausfordernd. Trotz Bemühungen und kleinen Hilfestellungen haben die Studenten es nicht geschafft das Tutorial vollständig abzuschließen. Auch hier war die fehlende Programmiererfahrung eine Hürde.
Bei den Teilnehmenden mit technischem Hintergrund, bestehend aus drei Informatikstudenten (zwei im sechsten und einer im dritten Semester) sowie einem Fahrzeugtechnikstudenten im dritten Semester, verlief die Bearbeitung der Aufgaben deutlich reibungsloser. Alle Studenten äußerten, dass das Tutorial trotz einiger technischer Herausforderungen grundsätzlich gut bewältigbar sei. Besonders hervorgehoben wurde dabei, dass die Aufgaben zwar fordernd, aber mit gegebenenfalls gegenseitiger Unterstützung gut lösbar seien.
\singlespacing
Da wir im Rahmen der THU Summer School überwiegend mit Studierenden mit technischem Hintergrund rechnen, entschieden wir uns bewusst dafür, die Schwierigkeit des Tutorials nicht weiter zu reduzieren. Die Möglichkeit der Zusammenarbeit unter den Teilnehmerinnen und Teilnehmern soll eventuelle Hürden abmildern und gleichzeitig die Teamarbeit fördern. Auf Basis des Feedbacks aus den Testläufen entschieden wir uns außerdem, das Lehrmaterial in Form einer PowerPoint-Präsentation aufzubereiten. Diese Entscheidung stützt sich sowohl auf die Unterlagen unseres Auftraggebers aus der vergangenen Summer School als auch auf die Rückmeldungen der testenden Studierenden, die die strukturierte und übersichtliche Aufbereitung in PowerPoint-Form positiv bewerteten.
\subsection{Fehlerquellen und potentielle Optimierungen}
Das Hauptergebnis unseres Projekts ist die Entwicklung eines strukturierten Hands-On Tutorials zur Gesichts- und Schrifterkennung auf einem Raspberry Pi. 
Im Gegensatz zu klassischen Projekten, bei denen ein fertiges Produkt im Vordergrund steht, handelt es sich bei unserem Projekt um die Erstellung von Lehrmaterialien, deren tatsächlicher Erfolg erst im Einsatz durch die Studierenden der THU Summer School sichtbar wird. Zur vorläufigen Evaluation des Materials führten wir, wie bereits beschrieben erste Testläufe mit Studierenden unterschiedlicher technischer Hintergründe durch. Eine abschließende Bewertung hinsichtlich Verständlichkeit, Schwierigkeitsgrad und Lerneffekt des Tutorials kann jedoch erst erfolgen, wenn es im Rahmen der Summer School eingesetzt und von einer größeren Gruppe internationaler Studierender bearbeitet worden ist. Wir erwarten, dass das Lernmaterial den Lernprozess unterstützt. 
 